\documentclass[10pt,landscape]{article}
\usepackage{multicol}
\usepackage{calc}
\usepackage{ifthen}
\usepackage[landscape]{geometry}
\usepackage{listings}

\usepackage{gitinfo2}

\def\Year{\expandafter\YEAR\the\year}
\def\YEAR#1#2#3#4{#3#4}

\usepackage[
    type={CC},
    modifier={by-nc-sa},
    version={3.0},
]{doclicense}

\usepackage{pdftexcmds}
\usepackage{catchfile}
\usepackage{ifluatex}
\usepackage{ifplatform}

\usepackage{emoji}

\ifmacosx
\setemojifont{Apple Color Emoji}
\fi

\iflinux
\setemojifont{Noto Color Emoji}
\fi


% To make this come out properly in landscape mode, do one of the following
% 1.
%  pdflatex latexsheet.tex
%
% 2.
%  latex latexsheet.tex
%  dvips -P pdf  -t landscape latexsheet.dvi
%  ps2pdf latexsheet.ps


% If you're reading this, be prepared for confusion.  Making this was
% a learning experience for me, and it shows.  Much of the placement
% was hacked in; if you make it better, let me know...


% 2008-04
% Changed page margin code to use the geometry package. Also added code for
% conditional page margins, depending on paper size. Thanks to Uwe Ziegenhagen
% for the suggestions.

% 2006-08
% Made changes based on suggestions from Gene Cooperman. <gene at ccs.neu.edu>


% To Do:
% \listoffigures \listoftables
% \setcounter{secnumdepth}{0}


% This sets page margins to .5 inch if using letter paper, and to 1cm
% if using A4 paper. (This probably isn't strictly necessary.)
% If using another size paper, use default 1cm margins.
\ifthenelse{\lengthtest { \paperwidth = 11in}}
	{ \geometry{top=.5in,left=.5in,right=.5in,bottom=.5in} }
	{\ifthenelse{ \lengthtest{ \paperwidth = 297mm}}
		{\geometry{top=1cm,left=1cm,right=1cm,bottom=1cm} }
		{\geometry{top=1cm,left=1cm,right=1cm,bottom=1cm} }
	}

% Turn off header and footer
\pagestyle{empty}
 

% Redefine section commands to use less space
\makeatletter
\renewcommand{\section}{\@startsection{section}{1}{0mm}%
                                {-1ex plus -.5ex minus -.2ex}%
                                {0.5ex plus .2ex}%x
                                {\normalfont\large\bfseries}}
\renewcommand{\subsection}{\@startsection{subsection}{2}{0mm}%
                                {-1explus -.5ex minus -.2ex}%
                                {0.5ex plus .2ex}%
                                {\normalfont\normalsize\bfseries}}
\renewcommand{\subsubsection}{\@startsection{subsubsection}{3}{0mm}%
                                {-1ex plus -.5ex minus -.2ex}%
                                {1ex plus .2ex}%
                                {\normalfont\small\bfseries}}
\makeatother

% Define BibTeX command
\def\BibTeX{{\rm B\kern-.05em{\sc i\kern-.025em b}\kern-.08em
    T\kern-.1667em\lower.7ex\hbox{E}\kern-.125emX}}

% Don't print section numbers
\setcounter{secnumdepth}{0}


\setlength{\parindent}{0pt}
\setlength{\parskip}{0pt plus 0.5ex}

\usepackage{xcolor}

\lstdefinestyle{mystyle}{
backgroundcolor=\color{black!100},  
basicstyle=\ttfamily\footnotesize\color{white},
numbers=left,                    
numbersep=5pt, 
tabsize=2,
numberstyle=\tiny\color{gray},
stringstyle=\color{white},
commentstyle=\color{white},
keywordstyle=\color{white},
showspaces=false,
showstringspaces=false
}

% -----------------------------------------------------------------------

\newcommand{\bash}[1]{\lstinline[language=bash]{#1}}

\begin{document}

\raggedright
\footnotesize
\begin{multicols}{3}



% multicol parameters
% These lengths are set only within the two main columns
%\setlength{\columnseprule}{0.25pt}
\setlength{\premulticols}{1pt}
\setlength{\postmulticols}{1pt}
\setlength{\multicolsep}{1pt}
\setlength{\columnsep}{2pt}

\begin{center}
     \Large{\textbf{Spack Cheat Sheet}} \\
\end{center}

\section{Directories}
\begin{tabular}{@{}ll@{}}
\bash{$tempdir/$user/spack-stage}    &  Built packages \\
\bash{$spack/var/spack/cache} & Source files \\
\bash{$spack/etc/spack/defaults/config.yaml} & Default configuration \\
\bash{$home/.spack/packages.yaml} & List of compilers \\
\bash{$home/.spack/packages.yaml} & External packages
\end{tabular}
The environment variable \bash{$user} is your username (moniker) and \bash{$tempdir} is the default temporary directory (usually \bash{tmp}. the default temporary directory can be changed by setting \bash{$TMP} pr \bash{$TMPDIR}.\\
\begin{tabular}{@{}ll@{}}
\bash{spack clean --stage}    &  Clean compiled \\
\bash{spack clean -a} & packages \\
\end{tabular}


\subsection{Installation and activation}
\newlength{\MyLen}
\settowidth{\MyLen}{\texttt{letterpaper}/\texttt{a4paper} \ }
\begin{lstlisting}[language=bash,style=mystyle]
# Install Spack
git clone --branch=releases/v0.22 
	https://github.com/spack/spack.git
# Activate Spack
. ./spack/share/spack/setup-env.sh
# Run space
spack
\end{lstlisting}
Please update \bash{v.022} with the latest release number. If you use a different
shell and not bash change \bash{setup-env.sh} to your favorite bash.


\subsection{Installation}
\settowidth{\MyLen}{\texttt{multicol} }
\begin{tabular}{@{}ll@{}}
\bash{spack info --phased}    &  Build details \\
\bash{spack install <name>} & Install package <name> \\
\bash{spack install -j n} & Number of $n$ parallel jobs  \\
\bash{spack install -v} & Verbose build output \\
\bash{spack install --no-cache} & Do not use cache \\
\bash{spack -v} & verbose output \\
\bash{spack -debug} & Debug output
\end{tabular}\\
To install a specific version of the package we use \bash{spack install wget@1.17} and to install a package with a specific compiler we use \bash{spack install wget \%gcc@12.2.0}. To specify the version and the compiler, we use \bash{spack install wget@1.17 \%gcc@13.2.0}.

\subsection{Loading and deleting packages}
\settowidth{\MyLen}{\texttt{multicol} }
\begin{tabular}{@{}ll@{}}
\bash{spack load <name>}    &  Load package <name> \\
\bash{spack load <name>@<version>} & Load package by <version> \\
\bash{spack load  /<hash>} & load package by <hash>  \\
\bash{spack info <name>} & List available versions \\ 
\bash{spack ilist <name>} & List packages \\ 
\bash{spack idelte <name>} & Delete package \\ 
\bash{spack  gc} & Garbage collection \\ 
\end{tabular}

\subsection{Adding compilers}
\settowidth{\MyLen}{\texttt{multicol} }
\begin{tabular}{@{}ll@{}}
\bash{spack compilers} & List all compilers \\ 
\bash{spack compiler find} & FInd new compilers \\ 
\bash{spack compiler info <name>} & Shows info 
\end{tabular} \\
If your favorite compiler is not available use \bash{spack install gcc@13.2.0}  or install \bash{spack install clang@17} to install it. That takes a while and you might want to get a hot beverage \emoji{coffee}.

\subsection{Adding external packages}
\settowidth{\MyLen}{\texttt{multicol} }
\begin{tabular}{@{}ll@{}}
\bash{spack external find openmpi} & Search for openmp \\ 
\bash{spack external find mpich} & Search for mpich \\ 
\bash{spack external find cuda} & Search for CUDA  \\
\bash{spack external find <name>} & Search for package <name>  \\
\end{tabular} \\
You can try to add system packages using \bash{spack external find python} or \bash{external find openssl}. Spack will use these packages and that can reduce the compilation time. 

\subsection{Concretization}
\settowidth{\MyLen}{\texttt{multicol} }
\begin{tabular}{@{}ll@{}}
\bash{spack install kokkos@4.4} & Install version 4.4. \\ 
\bash{spack install kokkos\%gcc@12} & Compile with gcc 12. \\ 
\bash{spack install kokkos +openmp} & Compile OpenMP backend \\ 
\bash{spack install kokkos ~openmp} & Do not compile OpenMP \\ 
\bash{spack install kokkos ^cuda@12.4.1} & Use CUDA 12.4.1 \\ 
\end{tabular} \\
To build Kokkos 4.4  with CUDA support using CUDA 12.4.1 support with gcc 12.2.0 we could use
\bash{spack install kokkos@4.4 +cuda ^cuda@12.4.1 \%gcc@12.2.0}.
To list all available variants, we use \bash{spack info <name>}.
\begin{lstlisting}[language=bash,style=mystyle]
spack info hdf5
Variants:
    cxx [false]                 false, true
        Enable C++ support
    fortran [false]             false, true
        Enable Fortran support
    mpi [true]                  false, true
        Enable MPI support
\end{lstlisting}
We observe that the package \bash{hdf5}  is build as default
\bash{spack install hdf5 ~cxx ~fortran +mpi}. The dependencies ($\wedge$ ) are listed as well.
\begin{lstlisting}[language=bash,style=mystyle]
Build Dependencies:
    cmake  gmake  java  mpi  mpich  ninja  szip  zlib-api

Link Dependencies:
    mpi  mpich  szip  zlib-api
\end{lstlisting}
To build the package with a specific version of  CMake we can use \bash{spack install hdf5 ^cmake@3.18}.

\subsection{Environments}
If a enviroment is activated all packages installed or deleted are related to the environment  and have no effect on other environment or the global spack installation.
\settowidth{\MyLen}{\texttt{multicol} }
\begin{tabular}{@{}ll@{}}
\bash{spack env create <name>} & Create env <name> \\ 
\bash{spack env activate <name>} & Activate env <name> \\ 
\bash{spack env deactivate} & Deactivate current env  \\ 
\bash{spack -e <env> install kokkos} & Install kokkos in env  \\
\end{tabular} \\
Create a named environment \emoji{smirk}
\begin{lstlisting}[language=bash,style=mystyle]
spack env activate -p foo
[foo] [diehlpk@darwin-fe1 ~]$ 
\end{lstlisting}
Show the activated environment \emoji{thinking}
\begin{lstlisting}[language=bash,style=mystyle]
spack env status
==> In environment foo
\end{lstlisting}

\subsection{Anonymous environments}
\begin{lstlisting}[language=bash,style=mystyle]
# Creating an anonymous environment 
spack env create --dir my_env
spack env create ./my_env
# Activate anonymous environment
spack env activate ./ my_env
\end{lstlisting}

\subsection{Development builds}
Spack will install all dependencies and run cmake to configure Kokkos but will not build Kokkos.
\begin{lstlisting}[language=bash,style=mystyle]
git clone https://github.com/kokkos/kokkos.git
cd kokkos
spack dev-build --fresh --drop-in bash \
	--until cmake kokkos@4.4 +cuda ^cuda@12.4.1 +openmp
cd spack-build
make -j 8
\end{lstlisting}
Note for development build need a specific version and Spack will not use the default one.






\rule{0.3\linewidth}{0.25pt}
\scriptsize\\
\doclicenseImage[imagewidth=3em] 20\Year\ Patrick Diehl \textit{\gitAbbrevHash} Version \gitRels




\end{multicols}
\end{document}
