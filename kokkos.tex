\documentclass[10pt,landscape]{article}
\usepackage{multicol}
\usepackage{calc}
\usepackage{ifthen}
\usepackage[landscape]{geometry}
\usepackage{listings}

\usepackage{gitinfo2}

\def\Year{\expandafter\YEAR\the\year}
\def\YEAR#1#2#3#4{#3#4}

\usepackage[
    type={CC},
    modifier={by-nc-sa},
    version={3.0},
]{doclicense}

\usepackage{pdftexcmds}
\usepackage{catchfile}
\usepackage{ifluatex}
\usepackage{ifplatform}

\usepackage{emoji}

\ifmacosx
\setemojifont{Apple Color Emoji}
\fi

\iflinux
\setemojifont{Noto Color Emoji}
\fi


% To make this come out properly in landscape mode, do one of the following
% 1.
%  pdflatex latexsheet.tex
%
% 2.
%  latex latexsheet.tex
%  dvips -P pdf  -t landscape latexsheet.dvi
%  ps2pdf latexsheet.ps


% If you're reading this, be prepared for confusion.  Making this was
% a learning experience for me, and it shows.  Much of the placement
% was hacked in; if you make it better, let me know...


% 2008-04
% Changed page margin code to use the geometry package. Also added code for
% conditional page margins, depending on paper size. Thanks to Uwe Ziegenhagen
% for the suggestions.

% 2006-08
% Made changes based on suggestions from Gene Cooperman. <gene at ccs.neu.edu>


% To Do:
% \listoffigures \listoftables
% \setcounter{secnumdepth}{0}


% This sets page margins to .5 inch if using letter paper, and to 1cm
% if using A4 paper. (This probably isn't strictly necessary.)
% If using another size paper, use default 1cm margins.
\ifthenelse{\lengthtest { \paperwidth = 11in}}
	{ \geometry{top=.5in,left=.5in,right=.5in,bottom=.5in} }
	{\ifthenelse{ \lengthtest{ \paperwidth = 297mm}}
		{\geometry{top=1cm,left=1cm,right=1cm,bottom=1cm} }
		{\geometry{top=1cm,left=1cm,right=1cm,bottom=1cm} }
	}

% Turn off header and footer
\pagestyle{empty}
 

% Redefine section commands to use less space
\makeatletter
\renewcommand{\section}{\@startsection{section}{1}{0mm}%
                                {-1ex plus -.5ex minus -.2ex}%
                                {0.5ex plus .2ex}%x
                                {\normalfont\large\bfseries}}
\renewcommand{\subsection}{\@startsection{subsection}{2}{0mm}%
                                {-1explus -.5ex minus -.2ex}%
                                {0.5ex plus .2ex}%
                                {\normalfont\normalsize\bfseries}}
\renewcommand{\subsubsection}{\@startsection{subsubsection}{3}{0mm}%
                                {-1ex plus -.5ex minus -.2ex}%
                                {1ex plus .2ex}%
                                {\normalfont\small\bfseries}}
\makeatother

% Define BibTeX command
\def\BibTeX{{\rm B\kern-.05em{\sc i\kern-.025em b}\kern-.08em
    T\kern-.1667em\lower.7ex\hbox{E}\kern-.125emX}}

% Don't print section numbers
\setcounter{secnumdepth}{0}


\setlength{\parindent}{0pt}
\setlength{\parskip}{0pt plus 0.5ex}

\usepackage{xcolor}

\lstdefinestyle{mystyle}{
backgroundcolor=\color{black!100},  
basicstyle=\ttfamily\footnotesize\color{white},
numbers=left,                    
numbersep=5pt, 
tabsize=2,
numberstyle=\tiny\color{gray},
stringstyle=\color{white},
commentstyle=\color{white},
keywordstyle=\color{white},
showspaces=false,
showstringspaces=false
}

% -----------------------------------------------------------------------

\newcommand{\bash}[1]{\lstinline[language=bash]{#1}}
\newcommand{\cpp}[1]{\lstinline[language=c++]{#1}}
\newcommand{\cppsign}{C\texttt{++}~}

\begin{document}

\raggedright
\footnotesize
\begin{multicols}{3}



% multicol parameters
% These lengths are set only within the two main columns
%\setlength{\columnseprule}{0.25pt}
\setlength{\premulticols}{1pt}
\setlength{\postmulticols}{1pt}
\setlength{\multicolsep}{1pt}
\setlength{\columnsep}{2pt}

\begin{center}
     \Large{\textbf{Kokkos Cheat Sheet}} \\
\end{center}

\newlength{\MyLen}
\settowidth{\MyLen}{\texttt{letterpaper}/\texttt{a4paper} \ }

\section{Installation}
\subsection{Spack}
\begin{lstlisting}[language=bash,style=mystyle]
. spack/share/spack/setup-env.sh
# Install with OpenMP execution space
spack install kokkos@4.4 +openmp %gcc@12.3.0
# Build with CUDA and OpenMP execution space
spack install kokkos +openmp +cuda cuda_arch=90
\end{lstlisting}
Note that the \bash{cuda_arch} needs to be updated to the architecture of your Nvidia GPU. You might want to update the Kokkos version to the latest release.
\subsection{CMake}
\begin{lstlisting}[language=bash,style=mystyle]
tar -xvf kokkos-4.4.01.tar.gz
cd kokkos-4.4.01
mkdir build
cd build
# Build with OpenMP execution space
cmake -DKokkos_ENABLE_OPENMP=ON ..
make
# Build with CUDA execution space
module load cuda/12.4.1
cmake -DKokkos_ENABLE_OPENMP=ON \
	-DKokkos_ENABLE_CUDA=ON \
	-DKokkos_ARCH_HOPPER90=ON .. 
make
\end{lstlisting}
Note that the CMake option \bash{Kokkos_ARCH_HOPPER90=ON} needs to be updated to the architecture of your Nvidia GPU. You can download the latest Kokkos realse
here \url{https://github.com/kokkos/kokkos/releases}.

\section{Initialization}

\begin{lstlisting}[language=c++,style=mystyle]
// Initialize Kokkos 
Kokkos::initialize (argc , argv );
{
    // Code
}
// Shutdown Kokkos
Kokkos::finalize ();
\end{lstlisting}
The header \cpp{#include <Kokkos_Core.hpp>} is required. It is recommend to place all Kokkos code within the curly parentheses. MPI needs to be initialized before Kokkos is initialized.
\section{Using Kokkos in CMake}
\begin{lstlisting}[language=bash,style=mystyle,escapechar=|]
$ vim CMakeLists.txt
cmake_minumum_required(VERSION 3.20)
# We need |\cppsign|since Kokkos is based on |\cppsign|
project(myKokkosProject CXX)
find_package (Kokkos REQUIRED)
add_executable (integration-kokkos integration-kokkos.cpp)
target_link_libraries (integration-kokkos 
	PRIVATE Kokkos::kokkos )
\end{lstlisting}

\section{Parallel algorithms}

\subsection{Parallel for loop}

\begin{lstlisting}[language=c++,style=mystyle]
size_t n = 1000;
Kokkos::parallel_for("Name",n, 
	KOKKOS_LAMBDA (const size_t i){
         // Code
     });
\end{lstlisting}

\subsection{Parallel reduction}
\begin{lstlisting}[language=c++,style=mystyle]
double sum = 0;
Kokkos::parallel_reduce ( "Name" ,n, 
	KOKKOS_LAMBDA  ( const size_t i , 
		double & valueToUpdate ) {
			valueToUpdate = // Do update
} ,sum );
\end{lstlisting}
We can use built-in reducers like \cpp{Kokkos::Max<double>(sum)} to obtain the maximum value.

\subsection{Nested parallelism}
\begin{lstlisting}[language=c++,style=mystyle]
Kokkos::parallel_for("Name", 
	Kokkos::MDRangePolicy<Kokkos::Rank<2>>({0, 0}, 
	{n, n})
 ,KOKKOS_LAMBDA(int64_t i, int64_t j) {
      // Do work });
\end{lstlisting}
Note you can have range policies up to six dimensions with Kokkos 4.
\section{Views}

\begin{lstlisting}[language=c++,style=mystyle]
size_t n = 1000;
// Vector
Kokkos::View<double*> vector("vector",n);
// Matrix
Kokkos::View<double**> matrix("matrix",n,n);
// Tensor
Kokkos::View<double***> tensor("matrix",n,n,n);
\end{lstlisting}

\subsection{Memory spaces}
The memory space is a template argument to a Kokkos view (\cpp{Kokkos::View<double*,Kokkos:HostSpace>}).
The following views are available:
\begin{itemize}
\item \cpp{Kokkos:HostSpace}
\item \cpp{Kokkos::CudaSpace}
\item \cpp{Kokkos::SharedSpace}
\end{itemize}
Note that the available space depends on the enabled Kokkos options during configuration. We have  \cpp{Kokkos::CudaSpace} because we enabled it.

\subsection{Memory layout}
\begin{lstlisting}[language=c++,style=mystyle]
Kokkos::View<double*, Kokkos::LayoutRight, x;
Kokkos::View<double*, Kokkos::LayoutLeft, y;
\end{lstlisting}
Default Layout: \cpp{Kokkos::LayoutLeft} for CudaSpace and \cpp{Kokkos::LayoutRight} for HostSpace.

\subsection{Mirror}
\subsubsection{Device to Host}
\begin{lstlisting}[language=c++,style=mystyle]
// Allocate the vectors within CUDA device memory
Kokkos::View<double*,Kokkos::CudaSpace> x =  Kokkos::View<double*,Kokkos::CudaSpace>("X",n);
// Generate host views of the vectors
Kokkos::View<double*,Kokkos::CudaSpace>
	::HostMirror x_host 
		= Kokkos::create_mirror_view(x);
// Fill vectors within the host memory
// Copy host data to device
Kokkos::deep_copy(x,x_host);
\end{lstlisting}

\subsubsection{Host to device}
\begin{lstlisting}[language=c++,style=mystyle]
// Create the mirror view
Kokkos::View<double*, Kokkos::HostSpace> 
	x_h("x_h",10000);
auto x = 
Kokkos::create_mirror_view(Kokkos::CudaSpace,x_h);
// Copy the data
Kokkos::deep_copy(x, x_h);
\end{lstlisting} 
To combine the creation of the mirror and the deep copy, one can use \texttt{Kokkos::create\_mirror\_view\_and\_copy ( Kokkos::CudaSpace, x\_h );}.

\subsection{Sub views}
\begin{lstlisting}[language=c++,style=mystyle]
Kokkos::View< double*** > t ( "T" , n , n , n);
auto slice = 
Kokkos::subview (T ,1,Kokkos::ALL,Kokkos::ALL);
\end{lstlisting} 

\begin{lstlisting}[language=c++,style=mystyle]
Kokkos::View< double** > t ( "T" , n , n  );
auto slice = 
Kokkos::subview (T ,0,Kokkos::make_pair(1,3));
\end{lstlisting} 

\section{Execution spaces}
\begin{lstlisting}[language=c++,style=mystyle]
size_t n = 1000;
Kokkos::parallel_for("Name",
	Kokkos::RangePolicy <Kokkos::Cuda>(0 , n ), 
	KOKKOS_LAMBDA (const size_t i){
         // Code
     });
\end{lstlisting}
Available execution spaces: \cpp{Kokkos::Cuda}, \cpp{Kokkos::Openmp}, and \cpp{Kokkos::Serial}. The available execution spaces depend on the options set a configuration.


\newpage

\section{Measure time}
\begin{lstlisting}[language=c++,style=mystyle]
// Start the timer
Kokkos::Timer timer;
//do work
// returns the elapsed time in seconds
std::cout << timer.seconds() << std::endl;
\end{lstlisting}

\section{Unmanaged memory}
\begin{lstlisting}[language=c++,style=mystyle]
std::vector<double> data(1000);
// Generate unmanned views from std::vector
Kokkos::View < double *, Kokkos::HostSpace > 
	x_h ( data.data(), data.size());
\end{lstlisting} 

\section{Scratch allocations}

\begin{lstlisting}[language=c++,style=mystyle]
// Allocate the memory
size_t n = 10000;
size_t m = 100000;
// Get size of scratch memeory
int scratch_size =
Kokkos::View <double**>
	::required_allocation_size(n , m);
// Allocate the memory
void * scratch =
	Kokkos::kokkos_malloc < >("Scratch" ,  scratch_size );
// Generate the views
Kokkos::View <double**> a_scr (scratch , n , m );
Kokkos::View <int*> b_scr (scratch , n );
\end{lstlisting}

\section{Atomic operations}

\begin{lstlisting}[language=c++,style=mystyle]
Kokkos::View<int*> sum("Sum", 1);
double x = 1;

// Parallel search
Kokkos::parallel_for("Search", n, 
	KOKKOS_LAMBDA(int64_t i) {
			if (v(i) == x) 
				Kokkos::atomic_add(&sum(0), 1);
});
std::cout << sum(0) << std::endl;
\end{lstlisting} 
Note that is just a showcase and should not be implemented like this.

\section{Memory traits}
\begin{itemize}
\item \cpp{Kokkos::MemoryTraits<Kokkos::Atomic>} -- All access to the view will be atomic.
 \item \cpp{Kokkos::MemoryTraits<Kokkos::RandomAccess>} -- Fast random access to read only memory on the GPU. Originally, used for textures in graphic programming.  
 \item \cpp{Kokkos::MemoryTraits<Kokkos::Unmanaged>} -- Convert a raw pointer to a Kokkos view
 \begin{lstlisting}[language=c++,style=mystyle]
const size_t n = 10000 ;
double* x_raw = malloc(n*sizeof (double));
// Unmanned view (No Label and 
// Kokkos does not deallocate the view)
Kokkos::View<double*, Kokkos::HostSpace, 
	Kokkos::MemoryTraits<Kokkos::Unmanaged>>
    x_view (x_raw, n);
free (x_raw);
\end{lstlisting} 
 \end{itemize}

For more details about Kokkos, please head to \url{https://kokkos.org/kokkos-core-wiki/}.\\
\rule{0.3\linewidth}{0.25pt}
\scriptsize\\
\doclicenseImage[imagewidth=3em] 20\Year\ Patrick Diehl \textit{\gitAbbrevHash} Version 0.1




\end{multicols}
\end{document}
